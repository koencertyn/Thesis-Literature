\section~\ref{sec:policies}
Om de verschillende soorten requests die binnenkomen te kunnen afhandelen maken we gebruik van policies.
Deze \texttt{policyEngine} is verantwoordelijk voor het verdelen van deze requests.

Uit de sectie waar de verschillende clouds voorgesteld worden, kunnen we opmaken dat elke cloud specifieke eigenschappen heeft.
Deze eigenschappen gaan vooraf bepaald moeten worden om de een basis onderscheid te kunnen maken.

\subsection{Cloud eigenschappen}
\subsubsection{Eigenschappen}
Verschillende eigenschappen zijn nodig om elk cloudplatform individueel te kunnen voorstellen.
Dit zijn de belangrijkste eigenschappen :
\begin{itemize}
	\item Bootsnelheid
	\item Schaalbaarheid
	\item Datamanagement
	\item Encryption
\end{itemize}
Deze krijgen allemaal een waarde toegewezen tussen 0 en 5, respectievelijk zwak naar sterk of 0 en 1.

De bootsnelheid is belangrijk voor applicaties waarvan de berekentijd vrij klein is.
Het voordeel dat we zouden krijgen bij het uitbreiden naar de cloud zou dan volledig weg zijn.
Bootsnelheid zou maximaal een tiental procent mogen bedragen van de totale vereiste responstijd.

De schaalbaarheid heeft belang voor processen die veel rekenkracht vragen.

Datamanagement houdt in hoe een gegeven platform met data omgaat.
Hoe meer mogelijkheden er zijn om met data om te gaan, hoe hoger dit cijfer zal liggen.

Encryption spreekt normaalgezien voor zich.
Sommige cloudplatformen bieden een basisencryptie aan terwijl anderen dit niet doen.

\subsubsection{Formaat}
Om zeker te zijn dat de \texttt{policyEngine} de vooraf bepaalde cloud eigenschappen kan interpreteren, moeten we een uniform formaat afspreken waarin deze beschreven zullen worden.
Ik heb ervoor gekozen om dit te doen op een XML gebaseerde stijl omdat iedereen hier al mee gewerkt heeft.
Het uiteindelijke bestand\footnote{\texttt{(CloudConfig.conf)}} waarin alle cloud eigenschappen in zullen komen, zal dus het volgende formaat hebben :
<cloud>
	<name> CloudName </name>
	<boot> Value 0-5 </boot>
	<scale> Value 0-5 </scale>
	<enc> Value 0-1 </enc>
</cloud>

Met als voorbeeld Openshift :
<cloud>
	<name> Openshift </name>
	<boot> 4 </boot>
	<scale> 4 </scale>
	<enc> 1 </enc>
</cloud>

Uit een lijst van vorige eigenschappen kunnen we per request de beste cloud kiezen.

\subsection{Requestproperties}

Voor we de meest optimale cloud kunnen kiezen moeten we het type request ook analyseren.
Een request heeft een paar eigenschappen die we moeten bepalen om deze door onze \texttt{policyEngine} te laten analyseren.
\begin{itemize}
	\item Geschatte verwerkingstijd
	\item Type request
	\item Tarief
\end{itemize}
De geschatte verwerkingstijd is nodig om in te schatten wat de maximale bootsnelheid van de cloud mag zijn.
Deze schatting zal gebaseerd zijn op gelijkaardige processen die in het verleden zijn behandeld.

Het type request is ook van enorm belang.
Er is een groot verschil in policies tussen het enkel opslaan van data (video's, afbeeldingen, ..) of het uitvoeren van een complexe wetenschappelijke berekening.

De prijs hangt af van een paar verschillende elementen.  Een eerste is de kostprijs van de publieke cloud.  Deze kan vast of variabel zijn.
Indien voor het goedkoopste traject gekozen zal worden, gaat er geen forwarding naar de cloud plaatsvinden en zal het proces enkel lokaal draaien.
Dit is enkel van toepassing voor requests waarvan de aard computationeel gericht is.







