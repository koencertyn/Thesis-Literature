\Section{Cloudplatformen}~\ref{sec:Cloud}
Op dit moment zijn er drie verschillende soorten cloudplatformen beschikbaar elk met hun eigen voor- en nadelen.
De eerste soort zijn IaaS of \texttt{infrastructure as a service} platformen.

//TODO image overzicht 3 platformen

Omdat dit soort platform enkel de infrastructuur voorziet moet de gebruiker zelf zijn operating system, applications ed. moeten installeren.
Dit heeft als groot voordeel de uitbreidbaarheid en controle die de gebruiker nu zelf in de hand heeft.
Nadelig is de complexiteit van ontwikkelen die hierdoor naar omhoog gaat.\newline

Een tweede soort voorziet al reeds de installatie van operating system alsook van de gewenste programma's.
Dit soort cloudplatform noemt men een PaaS of \texttt{platform as a service}.
Het grootste voordeel van dit platform is dat de gebruiker zich niet moet bezighouden met het installeren van operatingsystemen, programma's ed.
De voordelen die hieraan verbonden zijn is dat de programmeur zich niets moet aantrekken van het achterliggend systeem en zich kan focussen op de applicatie.

De laatste soort, SaaS of \texttt{software as a service} biedt al reeds geïnstalleerde software aan die direct te gebruiken is.

In deze thesis ga ik gebruik maken van de PaaS platformen omdat deze al reeds voorgeconfigureerde systemen aanbied.
Er zijn natuurlijk ook verschillende PaaS platformen waaruit gekozen moet worden.
In de volgende onderdelen is een overzicht van de voornaamste platformen met hun voor- en nadelen.


\Section{Openshift}~\ref{sec:Openshift}
Openshift, ontworpen door redHat, is één van de vele cloudproviders die PaaS platformen aanbieden.
Het maakt gebruik van verschillende \texttt{nodes} die op de reeds bestaande infrastructuur draaien.
In deze \texttt{nodes} zitten verschillende \texttt{gears} die verantwoordelijk zijn voor het draaien van de applicatie.
De applicatie kan ook nog verschillende tools gebruiken, deze worden \texttt{cartridges} genoemd.
Het is dus mogelijk om eigen gemaakte software in deze cartridges te steken om zo mee te deployen.
Tijdens het deployen sluit de applicatie af om zo via Maven\footnote{TODO} alle nodige dependencies te installeren.
Deployen kan gedaan worden op verschillende mogelijkheden :
\begin{itemize}
	\item Deployment via GitHub
	\item Deployment via .war files
\end{itemize}
Het voordeel van Openshift is de eenvoudigheid van deployen alsook de verschillende mogelijkheden van talen die ondersteund worden.

\Section{Heroku}~\ref{sec:Heroku}

\Section{Google App Engine}~\ref{sec:GAE}